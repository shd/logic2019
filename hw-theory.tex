\documentclass[10pt,a4paper,oneside]{article}
\usepackage[utf8]{inputenc}
\usepackage[english,russian]{babel}
\usepackage{amsmath}
\usepackage{amsthm}
\usepackage{amssymb}
\usepackage{enumerate}
\usepackage{stmaryrd}
\usepackage{cmll}
\usepackage[left=2cm,right=2cm,top=2cm,bottom=2cm,bindingoffset=0cm]{geometry}
\usepackage{proof}
\newcommand{\gq}[1]{\texttt{<<}#1\texttt{>>}}
\newcommand{\ogq}[1]{\overline{\texttt{<<}#1\texttt{>>}}}
\begin{document}

\begin{center}{\Large\textsc{\textbf{Теоретические (``малые'') домашние задания}}}\\
             \it Математическая логика, ИТМО, М3234-М3239, весна 2019 года\end{center}

\section*{Домашнее задание №1: <<знакомство с исчислением высказываний>>}

\begin{enumerate}

\item Расставьте скобки:
\begin{enumerate}
\item $\alpha\to\alpha\to\neg\beta\vee\beta\with\neg\alpha\vee\neg\beta\to\alpha\with\alpha\to\alpha\vee\beta\vee\beta$
\end{enumerate}

\item Покажите следующие утверждения, построив полный вывод (в частности, если пользуетесь теоремой о дедукции --- 
раскройте все преобразования):
\begin{enumerate}
\item $\alpha\vee\beta \vdash \neg(\neg\alpha\with\neg\beta)$
\item $\alpha\with\beta \vdash \neg(\neg\alpha\vee\neg\beta)$
\item $\alpha\to\beta\to\gamma \vdash \alpha\with\beta\to\gamma$
\item $\alpha\with\beta\to\gamma \vdash \alpha\to\beta\to\gamma$
\item $\alpha,\neg\alpha \vdash \beta$
\end{enumerate}

\item Покажите следующие утверждения, построив полный вывод (за полный
ответ будет считаться доказательство пяти утверждений из списка):
\begin{enumerate}
\item $\gamma \vdash \alpha\to\gamma$
\item $\alpha,\beta \vdash \alpha\&\beta$
\item $\neg\alpha,\beta \vdash \neg(\alpha\&\beta)$
\item $\alpha,\neg\beta \vdash \neg(\alpha\&\beta)$
\item $\neg\alpha,\neg\beta \vdash \neg(\alpha\&\beta)$
\item $\alpha,\beta \vdash \alpha\vee\beta$
\item $\neg\alpha,\beta \vdash \alpha\vee\beta$
\item $\alpha,\neg\beta \vdash \alpha\vee\beta$
\item $\neg\alpha,\neg\beta \vdash \neg(\alpha\vee\beta)$
\item $\alpha,\beta \vdash \alpha\rightarrow\beta$
\item $\alpha,\neg\beta \vdash \neg(\alpha\rightarrow\beta)$
\item $\neg\alpha,\beta \vdash \alpha\rightarrow\beta$
\item $\neg\alpha,\neg\beta \vdash \alpha\rightarrow\beta$
\item $\neg\alpha \vdash \neg\alpha$
\item $\alpha \vdash \neg\neg\alpha$
\end{enumerate}
\end{enumerate}

\section*{Домашнее задание №2: <<исчисление высказываний>>}

\begin{enumerate}

\item (Теоремы о корректности и полноте) Пусть $\Gamma$ --- какой-то список
высказываний и пусть $\alpha$ --- высказывание.

\begin{enumerate}
\item Покажите, что $\Gamma\vdash\alpha$ влечёт $\Gamma\models\alpha$.
\item Покажите, что $\Gamma\models\alpha$ влечёт $\Gamma\vdash\alpha$.
\end{enumerate}

\item (Теорема Гливенко) Рассмотрим исчисление высказываний, в котором 10 схема аксиом 
(аксиома снятия двойного отрицания)
$$\neg\neg\alpha\rightarrow\alpha$$
заменена на следующую:
$$\alpha\rightarrow\neg\alpha\rightarrow\beta$$
Такой вариант исчисления высказываний назовём интуиционистским.
Будем писать $\Gamma \vdash_{\texttt{и}} \alpha$, если существует
вывод формулы $\alpha$ из гипотез $\Gamma$ в интуиционистском исчислении
высказываний. Если же вывод производится в классическом исчислении
(изученном на 1 и 2 занятиях), будем указывать это как $\Gamma \vdash_{\texttt{к}} \alpha$.

\begin{enumerate}
\item Покажите, что если $\Gamma \vdash_{\texttt{и}} \alpha$, то 
$\Gamma \vdash_{\texttt{к}} \alpha$.
\item Покажите, что если $\alpha$ --- аксиома ($1\dots 9$ схемы), то 
$\vdash_{\texttt{и}} \neg\neg\alpha$.
\item Покажите, что $\vdash_{\texttt{и}} \neg\neg(\neg\neg\alpha\rightarrow\alpha)$.
\item Покажите, что если $\vdash_{\texttt{и}} \neg\neg\alpha$ и $\vdash_{\texttt{и}} \neg\neg(\alpha\rightarrow\beta)$,
то $\vdash_{\texttt{и}} \neg\neg\beta$.
\item Покажите, что если $\vdash_{\texttt{к}} \alpha$, то $\vdash_{\texttt{и}} \neg\neg\alpha$ (теорема Гливенко).
\item Покажите, что если $\Gamma \vdash_{\texttt{к}} \alpha$, то $\Gamma \vdash_{\texttt{и}} \neg\neg\alpha$.
\item Назовём (классическое или интуиционистское) исчисление \emph{противоречивым}, 
если для любой формулы $\alpha$ выполнено $\vdash \alpha$.
Покажите, что формула $\alpha$ исчисления, такая, что $\vdash \alpha$ и $\vdash\neg\alpha$, 
существует тогда и только тогда, когда исчисление противоречиво.
\item Покажите, что если классическое исчисление высказываний противоречиво, то противоречиво и интуиционистское
исчисление высказываний.
\end{enumerate}

\end{enumerate}

\end{document}
